\documentclass[11pt,oneside,a4paper]{article}

\usepackage[italian]{babel}
\usepackage[utf8]{inputenc}
\usepackage{amssymb,amsmath}
\usepackage{amsfonts}
\usepackage[nothing]{algorithm}
\usepackage{algpseudocode}
\usepackage{url}
\usepackage{tabularx}
\usepackage{float}
\usepackage{rotating}
\usepackage{multirow}

\title{Progetto didattico - corso di Logistica}
\author{Andrea Maggiordomo \\ Informatica applicata - Università di Pisa}
\date{2012}

% INIZIO DOCUMENTO
\addto{\captionsitalian}{\renewcommand{\abstractname}{Progetto n.3}}
\floatname{algorithm}{Algoritmo}
\renewcommand{\algorithmicrequire}{\textbf{Input:}}
\renewcommand{\algorithmicensure}{\textbf{Output:}}

\newfloat{myf}{h}{myfdata}

\begin{document}
\maketitle

\begin{abstract}
Si consideri il problema di dover distribuire delle merci da un deposito ad un
insieme di clienti. Presso il deposito sono disponibili $M$ veicoli che possono
fare esattamente un viaggio ciascuno. I veicoli sono tutti uguali e hanno una 
capacità in peso e in volume. Ciascun cliente richiede una quantità di merce di 
peso e volume dati. Ogni cliente deve essere visitato esattamente una volta. Dato 
l'insieme di vertici che rappresentano il deposito ed i clienti, è disponibile una 
matrice delle distanze per ogni coppia di vertici. Il problema è quello di
pianificare i viaggi di consegna minimizzando il costo complessivo.
\end{abstract}

\section{Descrizione del problema}
Il problema preso in considerazione è una variazione del \emph{Capacitated
vehicle routing problem}, nel quale la merce è caratterizzata da due grandezze 
distinte: peso e volume.

Sia $U = \{1,\dots,N\}$ l'insieme dei clienti che devono essere serviti, $0$ il
deposito; questo problema può essere formulato sul grafo non orientato $G = (V,E)$
con $V = U \cup \{0\}$, $E = \{\{i,j\} : i,j \in V,\: i \neq j\}$.
Ad ogni cliente $i \in U$ sono associate le quantità $w_i$, $v_i$ che definiscono
rispettivamente il peso e il volume della merce richiesta. Presso il deposito è disponibile 
una flotta di $M$ veicoli, tutti di capacità $C_w$ per quanto riguarda il peso
e $C_v$ per quanto riguarda il volume.
Ad ogni arco del grafo $\{i,j\} \in E$ è inoltre associata la quantità
$c_{ij} \in \mathbb{N}$ che ne determina il costo di attraversamento.
É stato considerato il caso simmetrico del problema con distanze eculidee:
$\forall i,j \in V$, $c_{ij} = c_{ji}$ e $\forall i,j,k \in V$,
$c_{ij} + c_{jk} \geq c_{ik}$.
Ulteriori vincoli impongono che un cliente sia servito da un solo 
veicolo, e che lo stesso veicolo non possa effettuare più di un viaggio.
L'obiettivo è quello di servire le domande di tutti i clienti rispettando i vincoli operativi
menzionati e minimizzando il costo complessivo del percorso.

\subsection{Formulazione matematica}
La formulazione scelta è un'estensione della classica formulazione a due indici
del \emph{CVRP}. Ad ogni arco $\{i,j\} \in E$ è associata la variabile intera $x_{ij}$
che definisce quante volte l'arco $\{i,j\}$ viene percorso. Una soluzione consiste
in una copertura del grafo con $M$ cicli, tutti adiacenti al deposito
e tali da partizionare $U$ in $M$ sottoinsiemi disgiunti. Le richieste totali dei
clienti appartenenti ai singoli cicli (rotte) non devono inoltre eccedere le capacità imposte ai veicoli.

La formulazione proposta è la seguente (dove $S(i)$ indica l'insieme di nodi adiacenti ad $i$) 

\begin{align}
  \min& \sum_{\{i,j\} \in E} c_{ij}x_{ij} \\
  %
  & \sum_{j \in S(i)} x_{ij} = 2 \qquad \forall \: i \in U \\
  %
  & \sum_{j \in S(0)} x_{0j} = 2M \\
  %
  &\sum_{\substack{\{i,j\} \in E \\ i \in S, j \notin S}} x_{ij} \geq 2L(S) \qquad
    \forall \: S \subseteq U,\:|S| \geq 2 \\
  %
  %& x_{ij} \in \{0,1\} \qquad \forall \: \{i,j\} \in E, i,j \neq 0 \\
  & 0 \leq x_{ij} \leq 1 \qquad \forall \: \{i,j\} \in E,\: i,j \neq 0 \\
  %
  %& x_{0j} \in \{0,1,2\} \qquad \forall \: \{0,j\} \in E
  & 0 \leq x_{0j} \leq 2 \qquad \forall \: \{0,j\} \in E \\
  %
  & x_{ij} \in \mathbb{Z} \qquad \forall \: \{i,j\} \in E
\end{align}

I vincoli (2) assicurano che in soluzione ogni cliente abbia grado $2$, imponendo
l'appartenenza di ciascun nodo in $U$ ad una sola rotta.
Il vincolo (3) fissa a $2M$ il grado del deposito, di fatto imponendo
l'esistenza di $M$ rotte in soluzione.
I vincoli (4), detti vincoli di capacità (\emph{rounded capacity inequalities}),
definiscono una valutazione inferiore sul numero di archi incidenti su ogni possibile
taglio che separa il deposito dai clienti. In pratica definiscono, per ciascun possibile
sottoinsieme $S$ di clienti, una valutazione inferiore sul numero di rotte necessarie per
servirne le richieste, garantendo contemporaneamente che in soluzione non esista nessuna
componente staccata dal deposito (infatti $\forall \: S \subseteq U, \: L(S) \geq 1$).
La quantità $L(S)$ è definita come
\begin{equation*}
  L(S) = \max\{W(S),Q(S)\}
\end{equation*}
dove
\begin{equation*}
  W(S) = \left\lceil\frac{\sum_{i \in S} w_i}{C_w}\right\rceil, \quad
  Q(S) = \left\lceil\frac{\sum_{i \in S} v_i}{C_v}\right\rceil
\end{equation*}
e impone che il numero di rotte scelte per servire $S$ sia almeno pari alla minima
quantità di veicoli in grado di servirne le richieste senza eccedere le limitazioni dei
veicoli. In generale ogni arco è percorribile una sola volta, ma il vincolo (6) permette
di assegnare agli archi adiacenti al deposito $x_{0j}$ anche il valore 2: il significato di
tale assegnamento è una rotta che fa meta verso il solo cliente $j$ (in questo caso infatti
$j$ ha automaticamente grado 2 e non può incidere su nessun altro arco in soluzione).

\subsection{Istanze}
Le istanze del problema sono state generate a partire da quelle disponibili
per il \emph{CVRP} presso \url{http://www.coin-or.org/SYMPHONY/branchandcut/VRP/data/}.
I parametri relativi alla quantità di volume richiesta sono $C_v$ che identifica la quantità
massima di richiesta in volume servibile da un singolo veicolo, e $\text{\emph{tightness}} \in (0,1)$
che regola il rapporto tra la quantità totale di merce richiesta dai clienti e la capacità
dell'intera flotta di veicoli.

Le istanze sono state generate con assegnamenti casuali oppure vincolati ad un intorno
della frazione di peso richiesta dallo stesso cliente.

Gli assegnamenti casuali prevedono la distribuzione della quantità di volume richiesta in maniera random
attorno alla domanda media.

Gli assegnamenti correlati al peso sono calcolati con il seguente algoritmo:
il generatore prende in input un ulteriore parametro $\text{\emph{rangeCap}} \in [0,0.99)$
che regola lo scarto tra la frazione di domanda in peso e quella in volume
(relativamente alle capacità del veicolo) richiesta dallo specifico cliente.
La quantità di volume $v_i$ richiesta dal cliente $i$ è quindi calcolata come la frazione
di volume equivalente alla frazione di peso richiesta, variata con uno scarto casuale
nell'intervallo $(-\text{\emph{rangeCap}},\;\text{\emph{rangeCap}})$.

Una volta che le domande in volume sono assegnate ai clienti, i singoli valori vengono
aggiustati per rispettare la relazione
\begin{equation*}
  \frac{\sum_{i \in U} v_i}{M \cdot C_v} = \text{\emph{tightness}}
\end{equation*}
cercando di rimanere dentro la frazione specificata per ogni cliente.

\section{Metodi di risoluzione}
In questa sezione vengono presentati gli algoritmi utilizzati per risolvere il problema, mentre
una descrizione degli aspetti implementativi più interessanti è lasciata alla sezione successiva.

Il primo algoritmo utilizzato è un algoritmo metaeuristico (tabu search) relativamente semplice
e basato su alcune delle idee discusse da Gendrau, Hertz e Laporte in \cite{gend94}. Presenta alcuni aspetti
interessanti relativamente all'esplorazione dello spazio delle soluzioni ed al come vengono
tollerate soluzioni non ammissibili durante la ricerca locale, ma ha il difetto di dipendere da
alcune scelte casuali rivelandosi non troppo stabile nei risultati ottenuti al variare di tali
scelte.

Il secondo algoritmo utilizzato è invece un metodo esatto basato sulla risoluzione della
formulazione matematica. Il metodo utilizzato è di tipo branch and cut: si parte da un
rilassamento combinatorio continuo della formulazione e vengono aggiunti
dinamicamente alcuni dei vincoli non considerati nel problema iniziale. Un approccio di questo tipo per
risolvere il \emph{CVRP} (del quale il problema preso in esame è parente stretto) è discusso ad esempio in
\cite{augerat98} e \cite{ralphs03}.

\subsection{Tabu search}
Gli algoritmi di ricerca locale si basano sull'idea di vicinato, considerando soluzioni
simili ad una data soluzione iniziale.

Nel caso del problema in esame una soluzione $x$ non è altro che un insieme $R_x$ di rotte
(sequenze ordinate di clienti) individuate sul grafo associato al problema.
L'insieme delle soluzioni vicine (\emph{neighbourhood} - $N(R_x)$) è costituito
da tutte le soluzioni ottenute da $R_x$ rimuovendo un cliente da una particolare rotta $r \in R_x$
e reinserendolo in una delle rotte di $R_x$ (non necessariamente diversa da $r$) nella posizione
più conveniente.

Al fine di evitare che la ricerca si fossilizzi intorno ad una particolare soluzione, l'algoritmo
di ricerca tabu impedisce che una soluzione già visitata sia esaminata nuovamente nelle iterazioni
immediatamente successive, inserendo la soluzione vietata in una struttura ausiliaria detta
\emph{tabu-list} per un numero arbitrario di iterazioni.

Dal momento che la somma delle domande dei clienti presenti in una rotta può eccedere le capacità
del veicolo, è necessario ``guidare'' la ricerca locale verso la regione ammissibile del problema.
Questo viene fatto, analogamente a quanto discusso in \cite{gend94}, introducendo una funzione
obiettivo ausiliaria $c'$ che penalizzi le soluzioni non ammissibili permettendo comunque
all'algoritmo di esaminarle. Le soluzioni durante la ricerca locale vengono valutate con
la funzione obiettivo ausiliaria
\begin{equation*}
  c'x = cx + \sum_{r \in R_x}\alpha w_{over}(r) + \sum_{r \in R_x}\beta v_{over}(r)
    \quad \alpha,\beta > 0
\end{equation*}
dove $w_{over}(r)$ e $v_{over}(r)$ determinano rispettivamente l'eccesso di domanda
in peso e in volume nella rotta r. Ovviamente se $x$ è ammissibile $cx = c'x$,
mentre se $x$ non è ammissibile i moltiplicatori $\alpha$ e $\beta$ determinano
una penalità da aggiungere al valore della funzione obiettivo standard.
Se durante la ricerca l'algoritmo spende troppo tempo a valutare soluzioni non ammissibili, i
moltiplicatori possono essere aggiustati per cercare di ricondurre l'esplorazione verso la
regione ammissibile, mentre nel caso contrario possono essere decrementati per indurre la ricerca
ad allontanarsi da un minimo locale.

L'algoritmo utilizzato (\emph{TabuRoute}) prevede una fase iniziale nella quale viene
identificata una soluzione da utilizzare come punto di partenza nella ricerca, ed una seconda
fase che si occupa della ricerca vera e propria.

\subsubsection{Scelta della soluzione iniziale}

La soluzione iniziale è scelta da un insieme di soluzioni candidate individuate utilizzando
delle semplici euristiche. L'ammissibilità in questa fase non è necessaria e viene semplicemente
scelta la soluzione che minimizza il valore della funzione obiettivo $c'$ (valutata ponendo
$\alpha = \beta = 1$).

Una prima soluzione è identificata utilizzando l'euristica del cliente più vicino per costruire
$M-1$ rotte ammissibili ed un'ultima rotta che includa ogni cliente non servito dalle precedenti.

Successivamente, altre $\frac{\sqrt{N}}{2}$ soluzioni sono costruite a partire da cicli
hamiltoniani costruiti sull'insieme dei clienti impiegando ancora l'euristica del nodo
più vicino (scegliendo ogni volta un diverso cliente come nodo iniziale), ed eseguendo la
procedura \emph{2-Opt} sui cicli individuati. Una soluzione è ottenuta da un ciclo hamiltoniano
costruendo le rotte in maniera greedy, partendo dal nodo iniziale. Anche in questo caso l'ultima
rotta individuata può rendere la soluzione non ammissibile.

Alla fine di questa fase la soluzione $x_u$ che minimizza $c'$ viene scelta come punto
di partenza per la ricerca locale. Se durante questa prima fase si incontra una soluzione ammissibile,
questa viene salvata come $\bar{x}$, altrimenti si pone $\bar{x} = \emptyset$.

\subsubsection{Ricerca locale}
La procedura di ricerca tabù è descritta nell'algoritmo \ref{search}.

%SEARCH ALGORITHM
\begin{algorithm}
\caption{Search}
\label{search}

\begin{algorithmic}[1]
\Require {$t_{max}$, $p$, $q$, $\bar{x}$, $x_u$}

  \State $\alpha \gets \beta \gets 1.0$;
  \State $x_{iter} \gets x_u$;
  \While{meno di $t_{max}$ iterazioni senza migliorare $\bar{x}$ e $x_u$}
    \If{$x_{iter}$ non ammissibile (peso) nelle ultime $10$ iterazioni}
      \State $\alpha \gets 2\alpha$;
    \ElsIf{$x_{iter}$ ammissibile (peso) nelle ultime $10$ iterazioni}
      \State $\alpha \gets \alpha/2$;
    \EndIf
    \If{$x_{iter}$ non ammissibile (volume) nelle ultime $10$ iterazioni}
      \State $\beta \gets 2\beta$;
    \ElsIf{$x_{iter}$ ammissibile (volume) nelle ultime $10$ iterazioni}
      \State $\beta \gets \beta/2$;
    \EndIf

    \State $Z \gets q$ elementi di $U$ scelti casualmente;

    \While{$Z \neq \emptyset$}
      \State $i \gets next(Z)$;
      \State $R \gets$ insieme delle rotte dei $p$ clienti più vicini ad $i$;
      \State $r_i \gets$ rotta a cui appartiene $i$;
      \State $S \gets$ insieme delle soluzioni generate da $x_{iter}$ effettuando le mosse
             non vietate $(i,r) : r \in R$;
      \State $x_{next} \gets x \in S : c'x = min\{c'x' : x' \in S\}$;
      \If{$c'x_{next} < c'x_u$} \Comment{Aggiornamento delle soluzioni}
        \State $x_u \gets x_{next}$;
      \EndIf
      \If{$x_{next}$ ammissibile $\land$ $cx_{next} < c\bar{x}$}
        \State $\bar{x} \gets x_{next}$;
      \EndIf
      \State $x_{iter} \gets x_{next}$;
      \State Dichiara tabu per $\theta$ iterazioni la mossa $(i,r_i)$;
      \Comment{$\theta \in [5,10]$}
    \EndWhile
  \EndWhile

\end{algorithmic}
\end{algorithm}

Gli aspetti fondamentali riguardano la gestione della tabu list, quali soluzioni vicine vengono
esaminate ad ogni iterazione e la regolazione dei parametri usati nella funzione obiettivo $c'$.

Per quanto riguarda la tabu list, dal momento che il passaggio da una soluzione $x$ ad una vicina
comporta il semplice reinserimento di un nodo $i$ in una rotta $r_i \in R_x$, è naturale
immaginarla come una lista di mosse $(i,r_i)$ vietate. Nello specifico viene impedito che un
nodo spostato di recente torni troppo presto nella rotta in cui si trovava precedentemente.

Il numero di soluzioni vicine esaminate è determinato da due parametri, $p$ e $q$. Quando
un cliente viene valutato per il reinserimento, vengono prese in considerazione solo
le rotte a cui appartengono i $p$ clienti più vicini ad esso; $q$ stabilisce invece la cardinalità
dell'insieme $Z$ dei clienti candidati ad essere valutati per il reinserimento ad ogni iterazione.

La mossa più conveniente tra quelle esaminate è immediatamente implementata.
Se durante l'esecuzione vengono trovate soluzioni migliori di $\bar{x}$ o $x_u$, queste
vengono aggiornate.

La funzione obiettivo $c'$ viene ricalibrata quando l'ammissibilità (relativamente al peso o
al volume, i casi sono gestiti separatamente) della soluzione non cambia per 10 iterazioni
consecutive.

L'algoritmo di ricerca locale viene invocato due volte, una prima volta con $q_1=\min(N-1,5)$
ed una seconda volta con $q_2=N-1$, il parametro $p$ è invece posto sempre a $N-2$: la seconda
chiamata si propone di intensificare la ricerca intorno alla miglior soluzione trovata.
TabuRoute è descritto dall'algoritmo \ref{taburoute}.

% TABU SEARCH ALGORITHM
\begin{algorithm}
\caption{TabuRoute}
\label{taburoute}

\begin{algorithmic}[1]
\Require {$t_{max}$}
\Ensure{$\bar{x}$ soluzione euristica del problema}

  \State $I \gets$ insieme di rotte iniziali candidate;
  \State $\bar{x} \gets$ miglior soluzione ammissibile (se esiste) di $I$;
  \State $x_u \gets x_i \in I : \forall\; x_j \in I,\: x_j \neq x_i,\: c'x_j \geq c'x_i$;
  \State Search($t_{max}|U|$, $p$, $q_1$, $\bar{x}$, $x_u$);
  \State Search($M|U|$, $p$, $q_2$, $\bar{x}$, $x_u$);
  \State Esegui \emph{2-Opt} su $\bar{x}$;

\end{algorithmic}
\end{algorithm}

\subsection{Risoluzione della formulazione matematica}
La formulazione matematica proposta non è direttamente risolvibile, dal momento che il numero
di vincoli di capacità (necessari a renderla completa) cresce esponenzialmente
con la quantità di nodi presenti nel grafo. L'approccio utilizzato è stato quindi quello di generare
questi vincoli solo quando necessari, nel modo prescritto dagli algoritmi di tipo Branch-and-cut.

Nell'accezione tradizionale gli algoritmi di questo tipo lavorano sulla formulazione completa,
risolvendo il rilassamento continuo e individuando disuguaglianze valide (cioè verificate da
qualsiasi soluzione ammissibile intera) che escludano la soluzione frazionaria ottenuta - il
 problema di individuare tali disugiaglianze viene detto \emph{problema di separazione}. Le
disuguaglianze trovate vengono quindi aggiunte alla formulazione e viene risolto
il ``nuovo'' problema di programmazione lineare ottenuto. Questa procedura è ripetuta iterativamente
fino a quando non viene individuata una soluzione per la quale non è più possibile trovare disuguaglianze
valide violate (\emph{algoritmo dei piani di taglio}).
Se a questo punto la soluzione individuata non è intera si rende necessario ricorrere al
branching ed iniziare ad esplorare l'albero di enumerazione.

Relativamente al problema preso in esame, tuttavia, i vincoli di capacità non sono semplici
disuguaglianze valide ma fanno parte a tutti gli effetti della formulazione matematica. Generando
dinamicamente questi vincoli può succedere quindi di imbattersi in soluzioni intere non
ammissibili per il problema originale. Fortunatamente, in questo caso individuare eventuali
vincoli di capacità violati è estremamente facile e si riconduce sostanzialmente ad una visita del
grafo non orientato ricavato dalla soluzione.

\subsubsection{Separazione dei vincoli di capacità}
Per risolvere il problema di separazione sono state utilizzate alcune delle procedure euristiche
descritte in \cite{augerat98} e in \cite{ralphs03}. Questi algoritmi lavorano sul grafo pesato
e non orientato $\hat{G} = \{V,\hat{E}\}$ ricavato dalla soluzione parziale
$\hat{x}$, con $\hat{E} = \{\{i,j\} \in E : \hat{x}_{ij} > 0\}$ e con peso sugli archi
$d_{ij} = \hat{x}_{ij}$. Si ricorre talvolta anche al grafo $\hat{G}_0$ ottenuto da $\hat{G}$ rimuovendo
il deposito e tutti gli archi incidenti ad esso.

Dal momento che ogni soluzione deve essere connessa, prima di avviare le euristiche di separazione
si effettua una semplice visita del grafo residuo per assicurarsi che tutti i nodi siano visibili
dal deposito. In caso contrario, i nodi non raggiungibili inducono un vincolo di capacità
violato che viene aggiunto alla formulazione, e si procede immediatamente alla risoluzione del
nuovo LP. Questo primo passo garantisce che le euristiche descritte in seguito operino sempre
su un grafo connesso.

Indicando con $S$ un insieme di clienti e con $\delta(S)$ la somma dei pesi
degli archi incidenti sul taglio di $\hat{G}$ $(S, V \setminus S)$, osserviamo che $S$ viola
un vincolo di capacità se $\delta(S) < 2L(S)$.
Le euristiche di separazione descritte in seguito cercano insiemi di clienti di questo
tipo.

\paragraph{ConnectedComponents}
La procedura \emph{ConnectedComponents} descritta in \cite{ralphs03} inizia considerando
l'insieme $C$ delle componenti connesse massimali di $\hat{G}_0$. Se una componente $C_i$ viola
un particolare vincolo di capacità questo viene aggiunto all'insieme di disuguaglianze trovate,
altrimenti l'algoritmo procede iterativamente rimuovendo da $C_i$ un nodo $v$ tale da
lasciare invariata la valutazione inferiore sul numero di veicoli necessari a servirne le
richieste fino a quando trova un vincolo di capacità violato, oppure non esiste più nessun
nodo con queste caratteristiche. Il procedimento è ripetuto per ogni elemento di $C$.

Come già menzionato le euristiche di separazione sono invocate sempre su soluzioni connesse.
Nel caso di soluzione intera, le componenti connesse massimali di $\hat{G}_0$ sono esattamente
gli insiemi di clienti serviti dalle rotte individuate in soluzione. In questo caso l'euristica
\emph{ConnectedComponents} si può considerare un separatore ``esatto'' e si ha la garanzia
che ogni soluzione intera non ammissibile sia tagliata dalla visita (se non connessa) oppure
da questa euristica (se una qualche rotta viola i vincoli di capacità). Ogni soluzione intera
non separata è quindi sicuramente ammissibile per il problema originale.

%CONNECTED COMPONENTS ALGORITHM
\begin{algorithm}
\caption{ConnectedComponents}

\begin{algorithmic}[1]
\Require{$\hat{G}_0$ ricavato da $\hat{x}$}
\Ensure {$D$ insieme di disuguaglianze violate da $\hat{x}$}

  \State Seleziona l'insieme $C$ delle componenti connesse massimali di $\hat{G}_0$;
  \ForAll{$C_i \in C$}
    \If{$\delta(C_i) < 2L(C_i)$}
      \State Aggiungi a $D$ la disuguaglianza violata indotta da $C_i$;
    \Else
      \State Determina, se esiste, il nodo $v \in C_i$ che minimizza
             $\delta(C_i \setminus \{v\})$ e tale che $L(C_i) = L(C_i \setminus \{v\})$;
      \If{$v$ non esiste}
        \State Passa alla prossima componente;
      \Else
        \State $C_i \gets C_i \setminus \{v\}$;
        \State goto 3;
      \EndIf
    \EndIf
  \EndFor

\end{algorithmic}
\end{algorithm}

\paragraph{Shrinking}
%SHRINKING ALGORITHM
\begin{algorithm}
\caption{Shrinking}

\begin{algorithmic}[1]
\Require{$\hat{G}$ ricavato da $\hat{x}$}
\Ensure {$D$ insieme di disuguaglianze violate da $\hat{x}$}

  \While{$true$}
    \If{$\exists\; \{i,j\} \in \hat{E}$ tale che $S = \{i,j\}$, $\delta(S) < L(S)$}
      \State Aggiungi a $D$ la disuguaglianza violata da $S$;
    \EndIf
    \State Seleziona l'arco $\{i,j\}$ non adiacente al deposito che massimizza $w_{ij}$;
    \If{$\nexists\; \{i,j\} \lor w_{ij} < 1$}
      \State Break;
    \Else
      \State Fondi i nodi $i$ e $j$ in un unico supernodo;
      \State Aggiorna $\hat{G}$ e il peso degli archi $w$;
    \EndIf
  \EndWhile

\end{algorithmic}
\end{algorithm}
La procedura \emph{Shrinking}, anch'essa descritta in \cite{ralphs03},
considera invece gli archi attivi della soluzione frazionaria, cercando coppie di
nodi $\{i,j\}$ di $U$ (deposito escluso) che violino qualche vincolo di capacità in $\hat{G}$.
Una volta esaminate tutte le coppie, viene individuato l'arco $\{h,k\}$ non adiacente al
deposito di peso $d_{hk}$ massimo in $\hat{E}$: se $d_{hk} \geq 1$, $h$ e $k$ vengono fusi
in un unico supernodo $u$ con 
%$w_u = w_h+w_k$, $v_u = v_h+v_k$ e $d_{ui} = d_{hi}+d_{ki}$
\begin{equation*}
w_u = w_h+w_k, v_u = v_h+v_k,\; d_{ui} = d_{hi}+d_{ki} \quad\forall i \in S(h) \cup S(k).
\end{equation*}
L'algoritmo procede iterativamente cercando coppie di
(super)nodi che violino qualche vincolo di capacità e collassando un arco come
specificato in precedenza fino a quando 
tutti gli archi sono adiacenti al deposito (nel qual caso non è possibile fondere i due
estremi senza includere il deposito stesso) oppure hanno peso minore di 1.

\paragraph{GreedyRoundCap}
La procedura \emph{GreedyRoundCap} è analoga ad una delle euristiche greedy descritte in
\cite{augerat98}; inizia considerando l'insieme $S$ composto da
un singolo cliente e cerca di espanderlo aggiungendo iterativamente un nodo
$v$ che minimizzi il valore di $\delta(S \cup \{v\})$, controllando ad ogni incremento se
$S$ viola un vincolo di capacità. Il procedimento è ripetuto usando ciascun cliente come elemento
di partenza in $S$;

%GREEDY ROUND CAP ALGORITHM
\begin{algorithm}
\caption{GreedyRoundCap}

\begin{algorithmic}[1]
\Require{$\hat{G}$ ricavato da $\hat{x}$}
\Ensure {$D$ insieme di disuguaglianze violate da $\hat{x}$}

  \ForAll{$i \in U$}
    \State $S \gets \{i\}$;
    \Repeat
      \State Aggiungi a $S$ il nodo $v$ che minimizza $\delta(S \cup \{v\})$;
      \If{$S$ viola un vincolo di capacità}
        \State Aggiungi la disuguaglianza violata da $S$ a $D$;
      \EndIf
    \Until{Non è più possibile estendere $S$}
  \EndFor

\end{algorithmic}
\end{algorithm}

\section{Implementazione}
In questa sezione vengono discusse alcune delle scelte implementative effettuate, il codice sorgente
è scritto interamente in C++. La realizzazione dell'algoritmo metaeuristico è abbastanza lineare ed è
trattata brevemente; più attenzione è dedicata invece alla descrizione di come è stato
implementato il branch and cut.

Per risolvere i modelli matematici sono state utilizzate le librerie messe a disposizione dal
progetto COIN-OR (\url{http://www.coin-or.org/}).
Nello specifico sono state impiegate la libreria Clp per risolvere i problemi di programmazione
lineare, la libreria Cgl per interfacciare le euristiche di separazione implementate con i solutori
e la libreria Cbc che offre un solutore specifico di tipo branch and cut.

Del branch and cut sono state realizzate due implementazioni: una si appoggia
ai solutori solamente per risolvere i problemi di programmazione lineare, controllando
direttamente la strategia di visita dell'albero di enumerazione, la strategia di branching e la
generazione delle righe della formulazione. Una seconda implementazione affida invece il processo
di ricerca interamente a Cbc, limitandosi a fornire il generatore di tagli specializzato.

\subsection{TabuRoute}
Come anticipato l'implementazione dell'algoritmo TabuRoute non presenta aspetti particolarmente
interessanti.

Per quanto riguarda la tabu-list, visto che le mosse vietate possono essere denotate da coppie
$(i,r)$ viene naturale rappresentarla usando una matrice intera $T$ di N righe e M colonne
che ``etichetta'' le mosse con il numero di iterazioni rimanenti per le quali queste sono ancora da
considerare vietate. Stabilire se la mossa $(i,r)$ è vietata è corrisponde a controllare se
l'elemento $T_{ij}$ sia positivo.

Per il resto sono state utilizzate quasi esclusivamente strutture dati e algoritmi messi a
disposizione dalla Standard Template Library. Una soluzione (insieme di rotte) è rappresentata da M vettori
di interi (singole rotte, l'ordine degli elementi determina la sequenza con la quale i clienti sono visitati
a partire dal deposito). La selezione dei $p$ clienti più vicini ad un dato cliente $i$ è effettuata
ordinando l'insieme $U$ con la funzione \texttt{std::sort} utilizzando un operatore di
confronto specializzato per il quale $j<k \iff c_{ij} < c_{ik}$, e selezionando i primi 
$p$ elementi dell'insieme così ordinato.

\subsection{Branch and cut}
I separatori sono implementati estendendo la classe \texttt{CglCutGenerator} fornita dalla
libreria Cgl per renderli compatibili con i solutori Clp e Cbc. La generazione
delle disuguaglianze è delegata all'implementazione del metodo
%\begin{verbatim}
\texttt{generateCuts(OsiSolverInterface\&, OsiCuts\&, CglTreeInfo)}
%\end{verbatim}
. Il generatore ricava la soluzione corrente dall'oggetto \texttt{OsiSol\-verIn\-ter\-fa\-ce} e 
inserisce i tagli individuati nella struttura \texttt{OsiCuts}.

L'applicazione effettiva dei tagli individuati alla formulazione è responsabilità di chi invoca questo metodo.

\subsubsection{Implementazione esplicita dell'algoritmo}
L'implementazione esplicita dell'algoritmo branch and cut (algoritmo \ref{bac}) utilizza un
oggetto \texttt{OsiClpSolverInterface} per risolvere i problemi di programmazione lineare via via
generati a partire dal modello iniziale $\mathcal{P}$.
I sottoproblemi (nodi dell'albero di enumerazione) sono rappresentati come una lista
di vincoli associati ad ogni variabile; ogni vincolo è identificato da una coppia $\langle$\emph{upper bound},
\emph{lower bound}$\rangle$.

%BRANCH AND CUT ALGORITHM
\begin{algorithm}
\caption{Branch and cut}
\label{bac}

\begin{algorithmic}[1]
\Require{$(G,c,w,v,C_w,C_v), \mathcal{P}, \epsilon$}
\Ensure{$\bar{x}$ soluzione ottima del problema}

  \State $\bar{x} \gets Euristica(G,c,w,v,C_w,C_v)$;
  \State $z_{ub} \gets c\bar{x}$;
  \State $Q \gets \{\mathrm{X}\}$;
  \Repeat
    \State $\mathrm{X}_i \gets next(Q)$;
    \Repeat
      \State $\hat{x} \gets LP(\mathcal{P},\mathrm{X}_i)$;
      \If{$\hat{x} = \emptyset \lor c\hat{x} \geq z_{ub}$}
        \State Chiudi il nodo $\mathrm{X}_i$;
      \ElsIf{$\hat{x}$ è ammissibile per il problema originale}
        \State $\bar{x} \gets \hat{x}, \: z_{ub} \gets c\hat{x}$;
        \State Chiudi il nodo $\mathrm{X}_i$
      \Else
        \State $\mathcal{L} \gets Cut(\hat{x},\mathcal{P})$;
        \State $\mathcal{P} \gets \mathcal{P} \cup \mathcal{L}$;
      \EndIf
    \Until{$\mathrm{X}_i$ chiuso $\lor$ $\mathcal{L} = \emptyset$}
    \If{$\mathrm{X}_i$ non è chiuso}
      \State $Q \gets Q \cup Branch(\mathrm{X}_i,\hat{x})$;
    \EndIf
    \If{Tutti i nodi al livello corrente sono visitati}
      \State Aggiorna $z_{lb}$ con il più piccolo valore $c\hat{x}$ rilevato;
    \EndIf
  \Until{$Q = \emptyset$ $\lor$ $\frac{z_{ub}-z_{lb}}{z_{lb}} \leq \epsilon$}

\end{algorithmic}
\end{algorithm}

Il nodo radice dell'albero di enumerazione, nel quale i vincoli sulle variabili sono quelli
imposti dal problema iniziale, è indicato con $\mathrm{X}$. Quando l'algoritmo dei piani di taglio
termina generando una soluzione frazionaria $\hat{x}$, è necessario ricorrere al branching.
Supponiamo che la variabile candidata al branching (scelta nel modo descritto in seguito) sia $i$,
 il problema $\mathrm{X}$ viene decomposto nei due sottoproblemi $\mathrm{X}_{i1} =
\mathrm{X}_{i} \cup \{x_i \leq \lfloor \hat{x}_i \rfloor\}$ e $\mathrm{X}_{i2} = \mathrm{X}_{i}
\cup \{x_i \geq \lceil \hat{x}_i \rceil\}$. È intuitivo verificare come questa regola (applicata
ad ogni nodo per il quale sia necessario ricorrere al branching) partizioni lo spazio delle
soluzioni indotto da X in due sottoinsiemi disgiunti. Questo comporta che l'albero di
enumerazione esamini al caso peggiore ogni possibile soluzione del problema originale, garantendo
quindi la correttezza dell'algoritmo.

La visita dell'albero è eseguita in ampiezza. I nodi sono
memorizzati in due code, quella dei nodi al livello corrente e quella dei nodi al livello
successivo. Quando un livello è esaminato per intero la corrispondente coda si svuota, e il lower bound
globale $z_{lb}$ è aggiornato con il più piccolo lower bound individuato esaminando i nodi al livello
appena completato.

L'identificazione delle disuguaglianze violate è affidata al generatore descritto in precedenza: 
queste, come accennato, sono memorizzate in un oggetto \texttt{OsiCuts} e sono poi applicate al modello
contenuto nel solutore con il metodo \texttt{Osi\-Clp\-Solver\-In\-ter\-fa\-ce::ap\-plyCuts( OsiCuts \& )}.

Sottoproblemi per i quali il valore della funzione obiettivo ecceda l'upper bound trovato
o non esistano soluzioni vengono potati. Allo stesso modo vengono chiusi sottoproblemi per
i quali si riesca ad individuare una soluzione intera ammissibile.

\subsubsection{Cbc}
Come già discusso, la strategia di risoluzione del problema usa un approccio di tipo branch and
cut per generare dinamicamente la formulazione.

Il solutore messo a disposizione da Cbc nella classe \texttt{CbcModel} implementa un algoritmo branch and cut
che, nella configurazione standard, opera sul modello IP specificato sotto l'ipotesi che questo sia
completo, ed offre la possibilità di invocare algoritmi di separazione generici.

Oltre ad implementare un separatore specializzato per il problema, è necessario quindi configurare il
solutore affinchè sia compatibile con l'approccio considerato. In particolare, visto che alcune
soluzioni intere potrebbero essere tagliate, bisogna impedire al solutore di considerare qualsiasi
soluzione come ammissibile prima che questa sia esaminata dai separatori. Occorre inoltre
continuare a tagliare le soluzioni frazionarie fino a quando non è più possibile
individuare ulteriori vincoli violati.

Il primo obiettivo è raggiunto passando al solutore un oggetto \texttt{OsiBabSol\-ver} che
contiene informazioni supplementari sulla strategia di risoluzione da impiegare.

\begin{verbatim}
CbcModel model;
OsiBabSolver solverCharacteristics;
solverCharacteristics.setSolverType( 4 );
model.solver()->setAuxiliaryInfo( &solverCharacteristics );
\end{verbatim}
Qui \texttt{setSolverType( 4 )} informa il solutore che una soluzione intera potrebbe ancora
essere tagliata.

Per imporre che il separatore venga chiamato fino a quando questo non fallisce nel trovare tagli è
sufficiente l'istruzione
\begin{verbatim}
model.cutGenerator( i )->setMustCallAgain( true );
\end{verbatim}
che applica tale condizione all'i-esimo separatore aggiunto al modello. 

Oltre a queste istruzioni occorre anche vietare l'ammissibilità delle soluzioni trovate
durante l'esecuzione dello strong branching, dal momento che queste non sono mai processate
attraverso i separatori. %A tal proposito è possibile estendere la classe
%\texttt{CbcFeasibilityBase} per indurre Cbc a considerare le soluzioni individuate in questa fase
%come non ammissibili.
Nel codice questa funzionalità è implementata attraverso la classe
\texttt{CbcFeasibilityNoStrong}, passando al modello un oggetto di questo tipo con il
metodo \texttt{setFeasibility}.

La strategia di visita utilizzata è sostanzialmente una depth-first fino alla prima soluzione ammissibile
trovata, dopodiché questa varia dinamicamente tra depth-first e breadth-first a seconda di come evolve
l'albero di enumerazione.

\section{Risultati computazionali}
La tabella \ref{t_inst} riporta i parametri utilizzati per generare le istanze di
prova.

%INSTANCES TABLE
\begin{table}[h]
%\newcolumntype{L}{>{\raggedright\arraybackslash}X}
%\begin{tabularx}{\textwidth}{|L|L|r|r|r|}
\centering
\begin{tabular}{|l|r|r|r|c|}
\hline
\multicolumn{1}{|l|}{Istanza} &
\multicolumn{1}{l|}{Sorgente} &
\multicolumn{1}{c|}{$C_q$} &
\multicolumn{1}{c|}{Tightness} &
\multicolumn{1}{c|}{RangeCap} \\
\hline
n16-k8.data & P-n16-k8 & 50  & 0.94 & 0.25 \\
n31-k5.data & B-n31-k5 & 80  & 0.87 & 0.15 \\
n37-k5.data & A-n37-k5 & 100 & 0.91 & 0.30  \\
n44-k6.data & A-n44-k6 & 200 & 0.91 & 0.40 \\
n46-k7.data & A-n46-k7 & 100 & 0.88 & 0.30  \\
n51-k7.data & B-n51-k7 & 1000 & 0.92 & 0.35 \\
n57-k9.data & B-n57-k9 & 250 & 0.83 & 0.15 \\
n64-k9.data & B-n64-k9 & 400 & 0.85 & 0.35 \\
n22-k4.data & E-n22-k4 & 1000 & 0.95 & n/a (random) \\
n30-k3.data & E-n30-k3 & 1000 & 0.88 & n/a (random) \\
n34-k5.data & A-n34-k5 & 100 & 0.95 & n/a (random) \\
n38-k6.data & B-n38-k6 & 200 & 0.90 & n/a (random) \\
\hline
\end{tabular}
\caption{Istanze generate}\label{t_inst}
\end{table}

I costi associati agli archi sono calcolati come suggerito dalla libreria di istanze TSPLIB, arrotondando i
valori frazionari con la funzione \texttt{nint}.

Tutti i risultati sono stati ricavati eseguendo i test su Linux, la versione di Cbc usata è la 2.7.

I risultati dell'algoritmo TabuRoute sono riportati nella tabella \ref{t_res:meta}.
Il gap è espresso rispetto al valore ottimo del problema ricavato in fase di test.
Per quanto riguarda i parametri algoritmici, incrementare oltre una certa soglia il valore di $t_{max}$
non ha comportato miglioramenti rilevanti. La scelta di $q$ e $p$ è stata guidata dalla volontà di permettere
alla ricerca di poter esaminare diverse soluzioni (ogni cliente viene reinserito in ciascuna delle rotte
individuate) senza rallentare eccessivamente le singole iterazioni - valori di $q$ troppo elevati
nella prima invocazione di \emph{Search} hanno palesato questo effetto collaterale. È giusto menzionare che in
alcuni casi si sono verificati cicli che hanno di fatto stallato la ricerca.

%TABUROUTE RESULTS
\begin{table}
%\newcolumntype{L}{>{\raggedright\arraybackslash}X}
%\newcolumntype{R}{>{\raggedleft\arraybackslash}X}
%\begin{tabularx}{\textwidth}{|L|R|R|R|r|}
\centering
\begin{tabular}{|p{1.8cm}|r|r|r|r|}

\hline
Istanza &
\multicolumn{1}{c|}{$t_{max}$} &
\multicolumn{1}{c|}{$z_{ub}$} &
\multicolumn{1}{c|}{Tempo} &
\multicolumn{1}{c|}{Gap} \\
\hline
n16-k8 &250& 484 & 0 & 0  \\
n22-k4 &250& 412 & 2 & 0 \\
n30-k3 &250& 583 & 0 & 6\% \\
n31-k5 &250& 686 & 1 & 0.73\% \\
n34-k5 &250& 830 & 1 & 2.47\% \\
n37-k5 &250& 747 & 1 & 5.96\% \\
n38-k6 &250& 862 & 1 & 4.10\% \\
n44-k6 &250& 996 & 107 & 5.28\% \\ 
n46-k7 &250& 981 & 3 & 4.58\% \\
n51-k7 &250& 1189 & 3 & 10.91\% \\ 
n57-k9 &250& 1630 & 8 & 2\% \\
n64-k9 &250& 1052 & 65 & 22.18\% \\
\hline

\end{tabular}
\caption{Risultati TabuRoute (seed 12345)}\label{t_res:meta}
\end{table}

Relativamente ai tempi di esecuzione, non emerge alcun legame rilevante tra questi e la qualità delle
soluzioni fornite, evidenziando un comportamento erratico dell'algoritmo. Un esame (abbastanza superficiale,
ma sufficiente a chiarire questo aspetto) della dipendenza dalla sequenza pseudo-casuale considerata, mostrato
nella tabella \ref{t_var:meta}, sembra suffragare questa ipotesi: i valori ricavati possono essere
estremamente variabili.

%TABUROUTE VAR
\begin{table}[h]
%\newcolumntype{L}{>{\raggedright\arraybackslash}X}
%\newcolumntype{R}{>{\raggedleft\arraybackslash}X}
%\begin{tabularx}{\textwidth}{|L|R|R|R|}
\centering
\begin{tabular}{|p{1.8cm}|r|r|r|}

\hline
Istanza &
\multicolumn{1}{c|}{$z_{ub}$ migliore} &
\multicolumn{1}{c|}{$z_{ub}$ peggiore} &
\multicolumn{1}{c|}{Variazione} \\
\hline
%n30-k3 &583& 583 & 0\\
n37-k5 &722& 735  & 1.8\% \\
n38-k6 &855& 889& 3.97\% \\
n44-k6 &1028&1144& 11.28\% \\
n46-k7 &971& 982 & 1.13\% \\
n51-k7 &1092& 1572 & 43.95\% \\
n64-l9 &905&1080 & 19.34\% \\
\hline

\end{tabular}
\caption{Variazione nei risultati di TabuRoute (5 run con $t_{max}=100$)}\label{t_var:meta}
\end{table}

%BRANCH AND CUT RESULTS
\begin{table}[h]

\newcolumntype{L}{>{\raggedright\arraybackslash}X}
\newcolumntype{R}{>{\raggedleft\arraybackslash}X}
\begin{tabularx}{\textwidth}{|L|r|R|r|r|R|r|r|}

\hline
\multicolumn{1}{|c|}{\multirow{2}{*}{Istanza}} &
\multicolumn{3}{c|}{CbcModel B\&C} &
\multicolumn{3}{c|}{VrpSolver B\&C} &
\multicolumn{1}{c|}{\multirow{2}{*}{cut-off}} \\
 &
\multicolumn{1}{c}{$z_{ub}$} &
\multicolumn{1}{c}{Tempo} &
\multicolumn{1}{c|}{Gap} &
\multicolumn{1}{c}{$z_{ub}$} &
\multicolumn{1}{c}{Tempo} &
\multicolumn{1}{c|}{Gap} & \\
\hline
n16-k8 & 484 & 0 & 0 & 484 & 0 & 0 & 485 \\
n22-k4 & 412 & 6 & 0 & 412 & 12 & 0 & 502\\
n30-k3 & 550 & 12 & 0 & 550 & 65 & 0 & 584\\
n31-k5 & 681 & 1 & 0 & 681 & 25 & 0 & 687\\
n34-k5 & 810 & 19 & 0 & 815 & \underline{3600} & 2.26\% & 842 \\
n37-k5 & 705 & 6 & 0 & 705 & 5 & 0 & 721 \\
n38-k6 & 828 & 115 & 0 & 828 & \underline{3600} & 0.85\% & 856 \\
n44-k6 & 946 & 2206 & 0 & - & - & - & 1029 \\
n46-k7 & 938 & 843 & 0 & 940 & \underline{3600} & 1.18\% & 972 \\
n51-k7 & 1072 & 392 & 0 & 1072 & \underline{3600} & 0.37\% & 1093 \\
n57-k9 & 1598 & 10027 & 0 & - & - & - & 1616 \\
n64-k9 & 861 & 38 & 0 & 861 & 593 & 0 & 929 \\
\hline

\end{tabularx}
\caption{Risultati branch and cut}\label{t_res:bac}
\end{table}
La tabella \ref{t_res:bac} riporta invece i risultati ottenuti con il branch and cut, usando
sia l'implementazione di Cbc che quella proposta nel progetto; la prima valutazione superiore (indicata
nella colonna \emph{cut-off}) è
ottenuta invocando \emph{TabuRoute} con $t_{max}=50$, inizializzato con 5 semi diversi.

Un aspetto interessante è la comparazione tra le prestazione dell'implementazione
esplicita e quella di Cbc. Su istanze di dimensioni contenute l'implementazione esplicta si comporta in
maniera accettabile, risultando più lenta ma riuscendo comunque a completare la visita dell'albero di
enumerazione -- eccetto il caso di n34-k5. Su istanze di dimensioni maggiori la chiusura del gap
si fa via via più problematica con l'eccezione dell'istanza n64-k9: in quest'ultimo caso la
rapidità con la quale si trovano ``buone'' soluzioni ammissibili permette di contenere le dimensioni
dell'albero di enumerazione e di convergere in tempi relativamente brevi alla soluzione ottima.

% CBC SEPARATORS COMPARISONS
\begin{sidewaystable}
\newcolumntype{C}{>{\centering\arraybackslash}X}
\newcolumntype{L}{>{\raggedright\arraybackslash}X}
\newcolumntype{R}{>{\raggedleft\arraybackslash}X}
\begin{tabularx}{\textwidth}{|l|rrr|rrr|rrr|rrr|}
\cline{2-13}

\multicolumn{1}{c|}{} &
\multicolumn{3}{C|}{\tt{CO=1 S=1}} &
\multicolumn{3}{C|}{\tt{CO=1 S=2}} &
\multicolumn{3}{C|}{\tt{CO=0 S=1}} &
\multicolumn{3}{C|}{\tt{CO=0 S=2}} \\

\hline
\multicolumn{1}{|l|}{Istanza} &
\multicolumn{1}{c}{\#Nodi} &
\multicolumn{1}{c}{Tempo} & 
\multicolumn{1}{c|}{Gap} &
\multicolumn{1}{c}{\#Nodi} &
\multicolumn{1}{c}{Tempo} & 
\multicolumn{1}{c|}{Gap} &
\multicolumn{1}{c}{\#Nodi} &
\multicolumn{1}{c}{Tempo} & 
\multicolumn{1}{c|}{Gap} &
\multicolumn{1}{c}{\#Nodi} &
\multicolumn{1}{c}{Tempo} &
\multicolumn{1}{c|}{Gap} \\

\hline
n16-k8 & 10 & 0.25 & -
       & 10 & 0.24 & -
       & 32 & 0.65 & -
       & 32 & 0.67 & - \\
n22-k4 & 1317 & 9 & - 
       & 1170 & 7 & -
       & 882 & 5 & -
       & 995 & 9 & - \\
n30-k3 & 2019 & 20 & -
       & 1358 & 12 & -
       & 2964 & 36 & -
       & 1393 & 12 & - \\
n31-k5 & 100 & 1 & -
       & 363 & 3 & -
       & 284 & 3 & -
       & 394 & 5 & - \\
n34-k5 & 1116 & 23 & -
       & 1249 & 32 & -
       & 744 & 14 & -
       & 852 & 18 & - \\
n37-k5 & 268 & 6 & -
       & 249 & 7 & -
       & 219 & 6 & -
       & 227 & 7 & - \\
n38-k6 & 2051 & 56 & -
       & 3526 & 126 & -
       & 1740 & 50 & -
       & 3154 & 109 &  -\\
n44-k6 & 21554 & \underline{3600} & 1.19\%
       & 18548 & \underline{3600} & 1.56\%
       & 9936 & 1040 & -
       & 18282 & 2704 & - \\
n46-k7 & 1757 & 95 & -
       & 7515 & 629 & -
       & 1378 & 71 & -
       & 1419 & 91 & - \\
n51-k7 & 19859 & 2577 & -
       & 4000 & 231 & -
       & 6696 & 603 & -
       & 2684 & 133 & - \\
n57-k9 & 17097 & 3744 & -
       & - & - & -
       & \underline{39000} & \underline{29932} & 4.89\%
       & - & - & - \\
n64-k9 & 448 & 40 & -
       & 446 & 40 & -
       & 446 & 40 & -
       & 446 & 44 & - \\

\hline

\end{tabularx}
\caption{Confronto tra varie combinazioni di separatori}\label{t_var:sep}
\end{sidewaystable}

\subsection{Configurazione del solutore general purpose}
% CBC CONFIG COMPARISONS
\begin{table}
\newcolumntype{C}{>{\centering\arraybackslash}X}
\newcolumntype{L}{>{\raggedright\arraybackslash}X}
\newcolumntype{R}{>{\raggedleft\arraybackslash}X}
\begin{tabularx}{\textwidth}{|l|rr|rr|rrr|}
\cline{2-8}

\multicolumn{1}{c|}{} &
\multicolumn{2}{c|}{\tt{CO=1 S=0}} &
\multicolumn{2}{c|}{\tt{CO=0 S=0}} &
\multicolumn{3}{c|}{\tt{CO=1 S=1}} \\

\hline
\multicolumn{1}{|L|}{Istanza} &
\multicolumn{1}{c}{\#Nodi} &
\multicolumn{1}{c|}{Tempo} & 
\multicolumn{1}{c}{\#Nodi} &
\multicolumn{1}{c|}{Tempo} & 
\multicolumn{1}{c}{\#Nodi} &
\multicolumn{1}{c}{Tempo} &
\multicolumn{1}{c|}{Gap} \\

\hline
n16-k8 & 12 & 0.17 
       & 47 & 0.74
       & 10 & 0.25 & -\\
n22-k4 & 1124 & 6  
       & 1148 & 7 
       & 1317 & 9 & -\\
n30-k3 & 1562 & 12 
       & 1684 & 15 
       & 2019 & 20 & -\\
n31-k5 & 94 & 1 
       & 290 & 4 
       & 100 & 1 & -\\
n34-k5 & 1297 & 19 
       & 916 & 13 
       & 1116 & 23 & -\\
n37-k5 & 275 & 6 
       & 233 & 6 
       & 268 & 6 & -\\
n38-k6 & 4212 & 114
       & 3323 & 91 
       & 2051 & 56 & -\\
n44-k6 & 18894 & 2209 
       & 24089 & 3350 
       & 21554 & \underline{3600} & 1.19\% \\
n46-k7 & 13019 & 848 
       & 1701 & 83 
       & 1757 & 95 & -\\
n51-k7 & 6630 & 394 
       & 9872 & 637 
       & 19859 & 2576 & -\\
n57-k9 & 30176 & 10027
       & 24267 & 5924 
       & 17097 & 3744 & -\\
n64-k9 & 450 & 38 
       & 450 & 39
       & 448 & 40 & -\\

\hline

\end{tabularx}
\caption{Confronto tra diverse configurazioni del solutore Cbc}\label{t_var:cbc}
\end{table}

Il solutore è stato provato in diverse configurazioni per valutare l'impatto che alcuni dei diversi
aspetti che caratterizzano un algoritmo branch and cut possono avere sulle prestazioni.
Nello specifico le possibilità esaminate sono state l'impiego o meno dei separatori generici implementati
nella libreria Cgl (tra i quali solo \texttt{CglProbing} e \texttt{CglGomory} si sono rivelati efficaci),
ed il passare al solutore una valutazione superiore valida sulla funzione obiettivo.

Nelle tabelle seguenti l'etichetta \texttt{CO} si riferisce all'imposizione della limitazione superiore (\texttt{1})
o meno (\texttt{0}) sulla funzione obiettivo, mentre \texttt{S} riguarda la combinazione
di separatori usata (\texttt{0} le sole euristiche, \texttt{1} ricorre ai tagli di Gomory se le
euristiche falliscono, \texttt{2} inizialmente invoca anche CglProbing per cercare di fissare qualche
variabile).

La tabella \ref{t_var:sep} riporta i risultati ottenuti impiegando i separatori generici. In maniera
inaspettata, l'uso di CglProbing si rivela controproducente in quasi
tutti i casi ed indipendentemente dall'imposizione del cut off sulla funzione obiettivo. Dai raffronti
la configurazione migliore sembra essere \texttt{[CO=1 S=1]}: la versione senza cut off si comporta
mediamente meglio ma evidenzia enormi difficoltà nella risoluzione dell'istanza n57-k9, mostrando come in
questo caso la disponibilità di un buon upper bound aiuti significativamente la risoluzione dell'istanza.

Il confronto con le configurazioni che fanno uso delle sole euristiche è riportato nella tabella
\ref{t_var:cbc}. Per quanto riguarda queste ultime la situazione è se possibile ancora meno definita, e
curiosamente la valutazione superiore che con i separatori generici risultava determinante per risolvere
l'istanza n57-k9 in questo caso comporta un decadimento delle prestazioni significativo.

Risulta estremamente difficile trarre conclusioni definitive su quale sia la configurazione
migliore tra quelle provate. L'utilizzo dell'upper bound sembra determinante in un solo caso,
mentre negli altri il contributo è molto meno evidente e diverse volte addirittura dannoso.
Questa scelta a livello teorico pare ragionevole, ed è stata operata con l'intento di contenere le
dimensioni dell'albero di enumerazione nel caso in cui l'algoritmo non riesca a trovare ``velocemente''
soluzioni ammissibili, ma con le istanze provate ciò non si è verificato in maniera consistente.
Similmente, anche l'impatto dei separatori generici è di difficile lettura e fortemente dipendente
dalle singole istanze, come evidenziato sempre dalla tabella \ref{t_var:cbc}. Interessante come
non emerga alcun nesso definito tra l'impiego dei separatori generici e una diminuzione del
numero di nodi esaminati dall'algoritmo che anzi talvolta risulta maggiore. Questo fatto può
essere in parte giustificato con l'ipotesi che le soluzioni intere a cui si converge con l'ausilio dei
separatori generici si rivelino spesso non ammissibili e vengano pertanto tagliate, oppure che il contributo
dei separatori alla progressione del lower bound non sia significativo.

\subsection{Prove supplementari}

In questa sezione sono riportate le prove effettuate per valutare l'impatto che i parametri relativi al
volume hanno sulla difficoltà delle istanze. Le prove sono state effettuate utilizzando il solutore Cbc in
configurazione \texttt{[CO=1 S=0]}, ovvero calcolando una limitazione superiore sulla funzione obiettivo e
ricorrendo alle sole euristiche specifiche per risolvere il problema di separazione. I tempi riportati
includono il calcolo dell'upper bound iniziale.

La tabella \ref{t_supp1} evidenzia le prove effettuate variando i rapporti tra le frazioni di peso e di
volume richieste dai clienti (ovvero modificando il parametro \emph{rangeCap}), mantenendo la quantità di
volume totale richiesta (regolata dal parametro \emph{tightness}) pari al 90\% della quantità massima
servibile. Le prove riportate nella tabella \ref{t_supp2} invece si riferiscono ad istanze nelle quali viene
variata la domanda totale dei clienti mantenendo costante il parametro \emph{rangeCap}. In questo secondo
caso è stato posto un tempo limite di 3600 secondi, per le istanze non risolte entro questo limite è
riportato il gap relativo.

% VOLUMEDEMANDRATIO=0.9
\begin{table}
\centering
\begin{tabular}{|l|r|r|r|}
\hline
\multicolumn{4}{|c|}{Tightness = 0.90} \\
\hline
Istanza sorgente & $\text{rc}=0.20$ & $\text{rc}=0.35$ & $\text{rc}=0.50$ \\
\hline

A-n32-k5.vrp & 9.71 & 11.73 & 7.55 \\
A-n33-k5.vrp & 9.09 & 9.42 & 4.76 \\
A-n33-k6.vrp & 7.16 & 9.40 & 104.97 \\
A-n34-k5.vrp & 9.44 & 6.54 & 8.01 \\
%A-n36-k5.vrp & 62.09 & ?* & 142.78 \\
A-n37-k5.vrp & 10.10 & 12.71 & 56.93 \\
%A-n38-k5.vrp & 0.96\% & 13.92 & 12.19 \\
A-n39-k5.vrp & 36.91 & 20.41 & 42.93 \\
A-n39-k6.vrp & 19.52 & 7.21 & 11.45 \\
A-n44-k6.vrp & 428.23 & 1172.78 & 20356.20 \\
A-n45-k6.vrp & 69.83 & 2134.80 & 262.80 \\
%A-n46-k7.vrp & 55.67 & 46.06 & -- \\
%A-n48-k7.vrp & -- & -- & -- \\
B-n31-k5.vrp & 5.54 & 55.18 & 3.26 \\
B-n35-k5.vrp & 3.58 & 135.14 & 54.86 \\
B-n38-k6.vrp & 5.57 & 9.49 & 5.17 \\
%B-n39-k5.vrp & 6.22 & 7.84 & 2.5\% \\
B-n41-k6.vrp & 13.57 & 11.08 & 230.61 \\
%B-n43-k6.vrp & 18.14 & -- & 15.25 \\
%B-n44-k7.vrp & 56.07 & 8.12\% & 1.97\% \\
B-n45-k5.vrp & 15.07 & 12.16 & 15.26 \\
%B-n45-k6.vrp & 46.72 & 80.99 & 4.16\% \\
B-n50-k7.vrp & 21.72 & 20.96 & 22.25 \\
%B-n50-k8.vrp & -- & -- & -- \\
%B-n51-k7.vrp & 53.27 & 31.29 & 1.92\% \\
E-n30-k3.vrp & 6.16 & 6.62 & 6.14 \\
E-n33-k4.vrp & 1.90 & 3.59 & 1.61 \\
%P-n19-k2.vrp & 0.22 & 0.22 & 0.22 \\
%P-n20-k2.vrp & 0.24 & 0.25 & 0.25 \\
%P-n21-k2.vrp & 0.71 & 0.72 & 0.62 \\
%P-n22-k2.vrp & 0.32 & 0.32 & 0.31 \\
%P-n22-k8.vrp & 31.64 & 400.68 & -- \\
%P-n23-k8.vrp & 59.69 & -- & -- \\
P-n40-k5.vrp & 7.91 & 6.51 & 7.64 \\
P-n45-k5.vrp & 18.09 & 191.18 & 16.02 \\
%P-n50-k7.vrp & 296.67 & 1.63\% & 1.73\% \\

\hline
\end{tabular}
\caption{Variazione RangeCap}\label{t_supp1}
\end{table}

In generale variare le proporzioni tra le richieste in un intervallo più ampio non si è rivelato un modo
particolarmente affidabile di ``complicare'' le istanze.
Più interessanti sono invece gli esperimenti effettuati variando la quantità
totale di domanda richiesta: in questo caso infatti la tendenza è più delineata. Nelle prove effettuate
con $\text{Tightness}=0.75$ molte istanze sono risolte con le stesse rotte della soluzione ottima del CVRP
standard, cioè per tale valore le quantità di volume richieste sono dominate in larga parte da quelle di
peso. Con valori più alti invece, le richieste relative al volume entrano in conflitto con quelle relative al
peso, invalidando un maggior numero di soluzioni e prolungando la procedura di ricerca.

% RANGECAP=0.5
\begin{table}
\centering
\begin{tabular}{|l|r|r|r|}
\hline
\multicolumn{4}{|c|}{RangeCap = 0.50} \\
\hline
Istanza sorgente & $\text{t}=0.75$ & $\text{t}=0.85$ & $\text{t}=0.95$ \\
\hline

A-n32-k5.vrp & 2.44 & 3.02 & 8.73 \\ 
A-n34-k5.vrp & 6.50 & 10.50 & 9.31 \\  
A-n36-k5.vrp & 11.67 & 431.83 & 19.32 \\  
A-n37-k5.vrp & 5.98 & 13.46 & 51.91 \\  
A-n38-k5.vrp & 11.38 & 17.96 & 9.90 \\  
A-n39-k5.vrp & 38.37 & 35.60 & 54.35 \\  
A-n39-k6.vrp & 24.69 & 38.71 & 121.55 \\  
A-n44-k6.vrp & 299.06 & 289.01 & \underline{8.39\%} \\  
%A-n45-k6.vrp & 151.81 & 675.75 & -- \\  
A-n46-k7.vrp & 24.58 & 136.34 & 1350.17 \\  
%A-n48-k7.vrp & -- & -- & -- \\  
B-n31-k5.vrp & 10.38 & 17.04 & 4.69 \\  
B-n34-k5.vrp & 4.31 & 3.32 & 14.47 \\  
B-n35-k5.vrp & 3.04 & 3.32 & 3.86 \\ 
B-n38-k6.vrp & 10.69 & 7.87 & 11.66 \\ 
B-n39-k5.vrp & 5.66 & 3.67 & 6.46 \\  
B-n41-k6.vrp & 10.67 & 24.04 & \underline{8.21\%} \\ 
B-n43-k6.vrp & 23.01 & 30.36 & 716.57 \\ 
B-n44-k7.vrp & 29.49 & 285.97 & 32.29 \\ 
B-n45-k5.vrp & 14.40 & 13.22 & 51.41 \\   
%B-n45-k6.vrp & 62.12 & 225.97 & -- \\   
B-n50-k7.vrp & 14.53 & 15.74 & 24.41 \\   
%B-n50-k8.vrp & -- & -- & -- \\  
%B-n51-k7.vrp & -- & -- & -- \\ 
E-n23-k3.vrp & 0.42 & 0.42 & 0.40 \\  
E-n30-k3.vrp & 6.21 & 2.79 & 7.88 \\  
E-n33-k4.vrp & 2.08 & 1.59 & 18.66 \\ 
%P-n20-k2.vrp & 0.24 & 0.24 & 0.33 \\ 
%P-n22-k2.vrp & 0.32 & 0.32 & 1.70 \\ 
%P-n22-k8.vrp & 5.49 & 1.70\% & -- \\  
%P-n23-k8.vrp & 777.61 & 609.09 & -- \\    
P-n45-k5.vrp & 19.01 & 19.87 & 44.85 \\  
P-n50-k7.vrp & 514.83 & 885.135 & 325.27 \\ 

\hline
\end{tabular}
\caption{Variazione Tightness}\label{t_supp2}
\end{table}

\section{Osservazioni}
Il contributo delle richieste di volume all'aumento della complessità del problema è evidente. Più 
la componente volume è ingombrante (sia nel rapporto tra domanda totale e domanda servibile, sia -- in
misura minore -- nella non correlazione tra i tipi di richieste) più le istanze diventano di difficile
risoluzione. Del resto un risultato di questo tipo potrebbe non essere troppo sorprendente: le richieste di
volume non fanno altro che porre un'ulteriore condizione sull'ammissibilità delle rotte.
Come evidenziato in \cite{ralphs03}, nel \emph{CVRP} i vincoli di capacità impongono un legame tra gli
instradamenti e la componente di \emph{packing} del problema; nel caso qui esaminato i vincoli di capacità
riguardano due dimensioni ``parallele'', e riducono ulteriormente l'insieme delle rotte ammissibili (quelle
compatibili con entrambe le richieste). La componente di \emph{routing} si interseca quindi non con una
ma con due diverse strutture di packing, aggiungendo un ulteriore grado di difficoltà alla risoluzione
del problema.

%RIFERIMENTI
\begin{thebibliography}{99}
\bibitem{gend94}
M. Gendreau, A. Hertz, G. Laporte: ``A Tabu Search Heuristic for the
Vehicle Routing Problem'', Management Science, Vol. 40, pp. 1276-1290, (1994)

\bibitem{augerat98}
P. Augerat, J. M. Belenguer, E. Benavent, A. Corberan, D. Naddef: ``Separating capacity
constraints in the CVRP using tabu Search'', European Journal of Operations Research 106, pp.
546–557, (1998)

\bibitem{ralphs03}
T. K. Ralphs, L. Kopman, W. R. Pulleyblank, L. E. Trotter Jr.:``On the Capacitated
Vehicle Routing Problem'', Mathematical Programming, Vol. 94, pp. 343-359, (2003)

\end{thebibliography}

\end{document}

% FINE DOCUMENTO
